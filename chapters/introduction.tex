%!TEX root = main.tex
\chapter{Introduction}\label{chapter:introduction}

Mankind always have been pursued to find their location and position in world correctly. By advancement of science and using of new electronic devices, this requirement have been resolved. Generally for locating, we need a reference point as origin or (0,0,0) to express the location of an object in every moment relative to this reference. This procedure will be more complicated when the desire object has pace. To estimate the position of moving object in each time unit, The difference of displacement, is calculated and updated.\\
Augmented Reality (AR) is a new technology that aims to generate a composite view for users. This view is a combination of the real view that user can see it and a virtual view such as graphics, sounds or animations which generates by computer. To augment the additional information to the real world, the geometry relation between the world and camera is necessary. These geometry relations that describes the position of camera relative to reference point in every moment is called tracking.\\
Tracking an object is a fundamental part of Augmented Reality (AR). Tracking means finding the location of an object or camera when they have movement in a sequence of frames relative to a reference point. Based on the AR application and degree of freedom of the object and the camera, there are two main tracking approaches:

\begin{itemize}
\item 2D Tracking: Estimate a 2D transformation which describes the 3D displacement of image projection of objects or a part of objects.
\item 3D Tracking: Identify the camera rotation and translation relative to the scene. It contains of 3 degrees of freedom for rotation and 3 degree for translation.
\end{itemize}

Due to the target applications and existence of so many mathematic approaches for solving the 3D tracking using a single camera, research in this field is substantially huge. marker-based and marker-less natural features-based techniques, are two methods to find out the position of camera or 3D objects tracking.\\

\section{Related Work}
\subsection{PTAM}
Georg Klein and David Murry \cite{klein2007parallel} proposed a method of estimating the pose (rotation and translation) of a camera without any prior knowledge about an small AR environment. The idea was adapted from SLAM algorithms in robotic with a novelty in implementation. Similar to Both SLAM and SFM, the whole procedure can be divided into two major tasks:
\begin{itemize}
\item Mapping: create a set of 3D feature points from environment that are seen from camera. This 3D World is used for increase the accuracy of tracking task.
\item Tracking: estimating the pose of camera with using the 3D map that are produced in previous step. In this approach the camera pose are computed by two levels. Estimating the initial pose and precise pose.
\end{itemize}
The key difference of PTAM algorithm compare to the simple SLAM and SFM is that uses of two parallel processing thread for executing the tracking and mapping tasks. This allows them to do this operation in the real-time.\\
Likewise, after each mapping task, the obtained 3D points are optimized by a bundle adjustment that is called local bundle adjustment. Furthermore, after a long iterations, a global bundle adjustment is used to refine the all points in 3D map. The result of PTAM compare to other approaches in this subject, is fast accurate and robust.

\subsection{Ubitrack Framework}
Ubitrack Framework is an open source framework for Augmented Reality. It was developed by Fachgebiet Augmented Reality chair (FAR) of the computer science faculty at Technical University of Munich.


\section{Motivation}
An accurate and robust estimation of camera pose is always a critical problem in some of computer vision applications. In augmented reality applications, two well-know techniques are used based on the knowledge that we have about the environment and type of applications: marker-based and feature-based approaches. As was introduced in previous section, The Ubitrack framework is an open source library for augmented reality application. The marker-based pose estimation was developed by Sven Barth as a master thesis in \cite{barth2014marker} and added to Ubitrack as the Feature Mapper module. This master thesis tries to implement an feature-based approach for Ubitrack with using the state-of-the-art algorithms for all tasks that are necessary for estimating the pose of monocular camera. Two robust feature matching and bundle adjustment modules also are implemented and added to Ubitrack that have a positive effect on our result.

\section {Approach}
In this thesis, we introduce a novel feature-based pose estimation approach which is implemented as the Feature Tracking module in Ubitarck. The first step of most of the feature-based techniques is matching (tracking) the feature points between two sequential frames $i$ and $i+1$. A robust and precise matching algorithm is caused to find the rotation and translation between these two frames. There are some techniques for matching the feature points but some of them are not enough fast and the other techniques suffer of some mismatch matching between two groups of feature points that is not good for our approach. To overcome this problem, a multi-layers feature matching is implemented that its compositional time is fast and its result is significantly good based on the ground truth data set.\\
In the next step, a new bundle adjustment is added to Ubitrack that is useful for optimizing the computed 3D points in pose estimation approach. Regarding to this fact that we use of both OpenCV and Ubitrack libraries, The cvsba bundle adjustment is selected because it was implemented by OpenCV data structure. It also support three types of optimization and describe in detail in bundle adjustment chapter.\\
Due to our experience for optimizing the 3D points in marker-based approach, we find that if we use of bundle adjustment after adding each new frame, the erratic image, bad matched feature points and the computed 3D points that are just visible in one or two frames have the negative effect on the result of bundle adjustment. For this purpose, for the first time a new concept is introduced that is grouping the images in same small package that is called bundle. Using the bundle concept and finding the feature points that are visible and available in all frames of a bundle, cause the error of bundle adjustment and 3D reconstruction decrease significantly. Similar to PTAM, we also use of two phases for estimating the pose of camera, mapping and tracking. Two levels of optimization, local bundle adjustment and global bundle adjustment, also are used to improve the accuracy of estimated 3D point in mapping phase.\\

This thesis document has the following structure. In the Chapter 2, the feature detection and extraction algorithms are explained. Next, the cvsba bundle adjustment and its usability in Ubitrack are introduced in Chapter 3. In Chapter 4, teh novel robust feature matching is described in detail. The bundle concept and our feature-based approach are described in the Chapter 5 and its result compare to the ground truth and marker-bade technique are shown in Chapter 6. Finally, in Chapter 7, conclusions are provided along with future work.


% In this master's thesis, we developed a new and novel approach rely on features-based to track the moving of a monocular camera. We assume the whole world is static except of camera. For the first phase of natural feature 3D tracking, feature points matching between each two sequence frames is critical and essential. An innovative method is developed, called robust feature matching, that extremely decrease the number of mismatching feature points. The other novelty of this master thesis that is unique and implemented for the first time is grouping the input images into the small size groups called bundles. Using of bundles (usually 5 frames) instead of a single frame increase significantly the accuracy of camera pose estimation. The result of this master's thesis is comparable with the state-of-the-art approaches in 3D tracking.\\

